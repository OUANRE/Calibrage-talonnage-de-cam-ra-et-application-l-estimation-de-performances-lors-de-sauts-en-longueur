% Chapter 2

\chapter{Implémentation} % Main chapter title

\label{Chapter2} % For referencing the chapter elsewhere, use \ref{Chapter1}

%----------------------------------------------------------------------------------------
%	Structuration du travail
%----------------------------------------------------------------------------------------
%\section{Structuration du travail} 

%\subsection{Feuille de route technologique}

%\subsection{Choix technologique}

\section{Calibrage et correction de la distorsion}
   L'étalonnage de la caméra est généralement 
   perturbé par l'angle, la lumière, l'équipement matériel, 
   etc. Il y a certaines erreurs matérielles et des défauts de conception, qui entraîneront différents degrés et types de distorsion du produit.
   Étant donné que cette recherche consiste à calculer la portée de l'image, 
   l'étalonnage de la caméra et la correction de la distorsion 
   doivent être effectués pour les images prise et pour la vidéo.\\
   La « méthode d'étalonnage de Zhang Zhengyou » est la plus courante et beaucoup utilisé.
      
\subsection{Etude de la méthode de Zhang}
   
   La méthode de calibrage proposé par zhang Zhengyou connue sous le nom méthode de calibrage de caméra basée sur le plan d'échiquier est une technique simple, flexible et réalisable à moindre coût.\\ 
   Cette technique proposée nécessite uniquement que la caméra observe un motif plan affiché dans quelques (au moins deux) orientations
   différentes. Le motif peut être imprimé et fixé sur une surface plane « raisonnable » (par exemple, une couverture de livre rigide).La caméra ou le motif planaire peuvent être déplacés à la main. L'approche proposée se situe entre le calibrage photogrammétrique et l'auto-calibrage car ils utilisent des informations métriques 2D plutôt que 3D ou purement implicites. \\
   
   Dans l'article le travail a été regroupé en quatre sections qui sont:\\
    
   \begin{itemize}
   	\item \textbf{La section 1 (Les contraintes de bases liées à l'observation d'un seul plan):} Comporte la notation de la matrice du caméras et les deux contraintes lié aux paramètres intrinsèques.\\
   	
   	\item \textbf{La section 2 (La procédure de calibrage):}  aborde d'abord la solution fermé qui permet de résoudre les contraintes liées aux paramètres intrinsèques , ensuite la technique d'optimisation non linéaire basée sur le critère du maximum de vraisemblance pour affiner les paramètres avec l'algorithme de Levenberg ­Marquardt et enfin la prise en compte de la distorsion radial de la lentille pour corriger la distorsion.\\
   	
   	\item \textbf{La section 4} étudie les configurations dans lesquelles la technique d'étalonnage proposée échoue\\
   	
   	\item \textbf{La section 5} fournie les résultats expérimentaux.
   \end{itemize}
   
   
   
   
   
%\section{Première partie: Mesure avec le bois}
%\section{Deuxième partie: Intervention d'humain }







