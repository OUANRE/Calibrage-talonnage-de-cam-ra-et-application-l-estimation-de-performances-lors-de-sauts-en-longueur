\chapter*{INTRODUCTION GÉNÉRALE}
\markboth{\MakeUppercase{INTRODUCTION GÉNÉRALE}}{}
%\addstarredchapter{INTRODUCTION GÉNÉRALES}
\addcontentsline{toc}{chapter}{INTRODUCTION GÉNÉRALE}
\adjustmtc
\thispagestyle{MyStyle}


  Dans les pays développés, l'intégration de différentes technologies dans le monde du sport a transformé la manière dont les compétitions sont jugées, les performances sont analysées et les entraînements sont conduits. Des outils sophistiqués comme la VAR (assistance vidéo à l'arbitrage), les capteurs biométriques, les systèmes de positionnement global (GPS) et la technologie de suivi des mouvements ont révolutionné le domaine sportif, apportant précision, objectivité et innovation.
  
  Cependant, ces technologies, bien que pratiques et innovantes, sont souvent très coûteuses, ce qui les rend inaccessibles à de nombreux pays en développement. De plus, leur maintenance nécessite une expertise technique continue, ce qui peut engendrer des frais supplémentaires et compliquer leur mise en œuvre dans des environnements aux ressources limitées.
  
  Au Burkina Faso, l'absence de ces technologies se fait sentir dans le domaine du sport. Pour surmonter ces défis et permettre au pays de bénéficier de technologies avancées tout en minimisant les coûts de maintenance et la nécessité de l'intervention constante de spécialistes, l'Université Norbert ZONGO nous a confié, dans le cadre de notre projet de fin de licence en informatique, la tâche de concevoir et d'implémenter un dispositif permettant d’estimer les performances lors des sauts en longueur.
  
  Notre projet se concentre sur le calibrage ou étalonnage de caméra, une technique cruciale pour obtenir des mesures précises à partir d'images et de vidéo. Le calibrage vise à déterminer avec précision les caractéristiques géométriques et optiques de la caméra, permettant ainsi de corriger les distorsions et d'assurer une interprétation fidèle des images capturées. En établissant une correspondance exacte entre les pixels de l'image et les dimensions réelles de l'espace filmé, le calibrage de caméra devient une étape fondamentale pour obtenir des données fiables et exploitables. Lorsqu'il est appliqué à l'estimation des performances athlétiques, et plus spécifiquement aux sauts en longueur, le calibrage de caméra permet de transformer les séquences vidéo en informations quantitatives précises. Il est ainsi possible de mesurer des paramètres clés tels que la longueur du saut, la vitesse de l'athlète, et les angles des différentes phases du mouvement. Ces données sont cruciales pour les entraîneurs et les chercheurs, car elles fournissent une base objective pour analyser les performances, identifier les points à améliorer, et développer des stratégies d'entraînement personnalisées.
  
  Ce dispositif vise non seulement à fournir des outils de mesure précis et accessibles, mais aussi à réduire la dépendance à des technologies coûteuses et à l'expertise technique externe. En somme, notre objectif est de développer une solution innovante et économique pour améliorer l'analyse et l'optimisation des performances sportives au Burkina Faso.
  
  Ainsi donc, ce document sera structuré comme suit :Dans le premier chapitre, intitulé 'État de l'art', nous aborderons les différents mots-clés qui composent le thème.
  et dans le deuxième chapitre, nous présenterons notre implémentation.
  
  
  
  
  
  
  