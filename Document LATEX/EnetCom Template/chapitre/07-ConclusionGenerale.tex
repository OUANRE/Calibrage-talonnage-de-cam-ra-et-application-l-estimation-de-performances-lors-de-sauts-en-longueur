\chapter*{CONCLUSION GÉNÉRALE}
\markboth{\MakeUppercase{CONCLUSION GÉNÉRALE}}{}
\addcontentsline{toc}{chapter}{CONCLUSION GÉNÉRALE}
\adjustmtc
\thispagestyle{MyStyle}


 Dans le but de combler le fossé technologique dans le domaine sportif au Burkina Faso, notre projet sur le calibrage et l'étalonnage de caméras, appliqué à l'estimation des performances lors des sauts en longueur, a souligné l'importance cruciale de la précision technologique dans l'analyse sportive. En explorant les aspects théoriques et pratiques du calibrage de caméras, nous avons développé un dispositif capable de mesurer la distance entre la ligne d'appel et le point d'atterrissage sur les images capturées par notre caméra. Pour ce faire, nous avons étudié différentes approches de calibrage, à savoir l'étalonnage par mire, l'auto-étalonnage et l'étalonnage hybride. Bien que toutes ces approches soient excellentes, nous avons opté pour l'étalonnage par mire, ce qui nous a permis d'utiliser la méthode de Zhang Zhengyou. Cet étalonnage s'est avéré très utile, car nous avons pu utiliser les résultats des différents paramètres pour convertir la distance en pixels en mètres.
 
 Les principaux défis rencontrés incluaient la complexité technique du calibrage, nécessitant une compréhension approfondie des caractéristiques géométriques et optiques des caméras, ainsi que la nécessité de développer des algorithmes sophistiqués pour corriger les distorsions et interpréter fidèlement les images capturées. De plus, l'écriture du programme de calcul de distance s'est avérée complexe, car il nous fallait entraîner un modèle de détection d'objets.
 
 Malgré ces défis, le projet a apporté de nombreux bénéfices. Il nous a permis d'acquérir une expertise précieuse en traitement d'images, tout en démontrant la faisabilité de créer des technologies avancées à faible coût. En plus d'acquérir de nouvelles connaissances dans un domaine qui nous était étranger, ce projet nous a permis de choisir le domaine dans lequel nous allons nous spécialiser.
 
 En somme, ce projet a non seulement renforcé nos compétences techniques et notre capacité à innover dans un environnement aux ressources limitées, mais il a également montré le potentiel d'une technologie accessible pour améliorer l'analyse et l'optimisation des performances sportives au Burkina Faso et dans d'autres pays en développement. Nous espérons que ce travail servira de base pour des initiatives futures, contribuant à la démocratisation de la technologie sportive à travers le monde.